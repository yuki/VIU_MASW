\newcommand{\ClassPath}{../VIU_TFM_LaTeX_template}
\documentclass{\ClassPath/viu-tfm-template}


\definecolor{maincolor}{HTML}{e65218}

%--------------------------------------------------------------------------
% Definiciones necesarias Modifica con tus datos
%--------------------------------------------------------------------------
\def\nombre{Gómez Olivencia, Rubén}
\def\dni{78910013-A}
\def\titulo{Análisis y diseño de la aplicación web \linebreak\linebreak\linebreak para la Asociación Internacional de Chefs}
\def\titulacion{Máster Universitario en Desarrollo de Aplicaciones y Servicios Web}
\def\curso{2022-2023}

%Los siguientes son opcionales: si no se ponen, la portada cambia un poco. Ideal para escribir artículos/trabajos cortos
\def\dirige{}
\def\convocatoria{}
\def\asignatura{Ingeniería Software Web}


% importar fichero de Bibliografía
\addbibresource{Actividad_2.bib}

\begin{document}
    \graphicspath{{../VIU_TFM_LaTeX_template/}}

    \coverpage

    \tableofcontents

\chapter{Introducción}
En el siguiente documento se va a detallar el camino utilizado para la realización de la aplicación web solicitada por la  \textbf{Asociación Internacional de Chefs (AIC)} para la gestión y compartición de recetas culinarias.

A lo largo del documento se diferenciarán distintos apartados en los que se detallarán cuál ha sido la metodología empleada durante todo el proceso, la planificación realizada y cómo se ha hecho la obtención y posterior análisis de requisitos.

Con todo ello, se ha realizado el modelado de datos que es necesario para llevar a cabo la petición del cliente, así como distintos diseños conceptuales que serán explicados en sus respectivos apartados.

Para finalizar se han incluido las conclusiones de todo el proceso llevado a cabo.


\chapter{Metodología utilizada}
Para llevar a cabo la creación de la aplicación web, y todo el proceso subyacente, se ha hecho uso del método “diseño centrado en el usuario” (\textbf{DCU}), y tal como nos dice la definición de \textcite{hassan2004diseno} “se caracteriza por asumir que todo el proceso de diseño y desarrollo del sitio web debe estar conducido por el usuario, sus necesidades, características y objetivos”.

Es por eso, que para llevar a buen puerto el proyecto encargado por la AIC, debemos de comprender las funcionalidades presentadas y sus requerimientos buscando como objetivo final el tener la mejor experiencia de usuario.

\chapter{Planificación}
Antes de comenzar con los requisitos del proyecto, y con el objetivo de tener un mejor entendimiento y comunicación con la \textbf{AIC}, se ha realizado una planificación para definir el alcance del proyecto y el léxico común que vamos a utilizar.

De esta manera, podemos decir, tal como se dice en el lenguaje coloquial, que estamos todos los involucrados en el proyecto (tanto el cliente \textbf{AIC}, como los analistas y desarrolladores que hemos llevado a cabo el proyecto) en la misma página.

No está de más recordar que el objetivo común es llevar a buen término el proyecto de creación de la aplicación web de recetas.

\section{Alcance del proyecto}
Aunque se detallarán más adelante de manera más granular los requisitos y funcionalidades, a continuación se expone la premisa principal del alcance del proyecto y cuál es el deseo por parte de \textbf{AIC} en este proyecto.

\begin{enumerate}[label=\alph*)]
    \item Registrar el usuario a la plataforma de la AIC.
    \item Autenticar usuario al ingresar a la aplicación de la AIC.
    \item Cargar la receta, con todos sus aspectos asociados (categoría, lista de
    ingredientes, procedimiento, dificultad, costo, etc.)
    \item Buscar recetas en base a distintos aspectos: clase de receta, tipo de ingrediente
    de base, nivel de dificultad, costo de la receta.
    \item Visualizar resultados de búsqueda.
    \item Visualizar la información sobre cada receta.
    \item Compartir receta en las redes.
    \item Valorar receta (de 1 a 5 estrellas).
\end{enumerate}

Todo ello se realizará bajo una aplicación en entorno web, cuyos usuarios objetivos son \textbf{personas adultas de toda Europa y América}.


Como ya se ha comentado, esto son los \textit{items} principales que se pretenden conseguir y que posteriormente serán analizados en detalle.


% TODO: meter léxico y/o glosario? \_(o_o)_/
%\section{Léxico asociado}



\chapter{Requisitos del cliente}

En este apartado vamos a detallar cómo se ha realizado toda la toma de requisitos y el análisis para poder realizar una diferenciación de los mismos para su posterior modelado, tanto de datos como de interfaces.

Antes de continuar, cabe recordar qué es un requisito, y tal como nos dice \textcite{Sommerville2005} “es una declaración abstracta de alto nivel de un servicio que debe proporcionar el sistema o una restricción de éste”.

Así mismo, y teniendo en cuenta la definición proporcionada por la \textcite{IEEE610} en su estándar 610.12-1990, se define como “una condición o capacidad que debe estar presente en un sistema o componentes de sistema para satisfacer un contrato, estándar, especificación u otro documento formal”.


\section{Obtención de los requisitos}
Para la obtención de los requisitos se realizaron una serie de entrevistas con los responsables de la \textbf{AIC} con el fin de recabar toda la información posible para que de esta manera quedasen claro las funcionalidades y requisitos mínimos que debe tener la aplicación web.

A la reunión acudieron los responsables de negocio de la Asociación Internacional de Chefs (\textbf{AIC}) y por otro lado, en función de analista, Rubén Gómez Olivencia.




\section{Análisis de requisitos}
Una vez realizado las reuniones con los responsables de \textbf{AIC} el paso llevado a cabo ha sido el de analizar toda la información recabada para pasar a formalizar los requisitos que debe cumplir el proyecto.

Este análisis ha sido diferenciado en las siguientes categorías:

\begin{itemize}
    \item Requisitos funcionales
    \item Requisitos no funcionales
\end{itemize}

En cada uno de los diferentes apartados de este documento se explicarán el alcance de cada una de las categorías.


\subsection{Requisitos funcionales}



\subsubsection{Requisitos de datos}


\subsubsection{Requisitos de presentación}


\subsubsection{Requisitos de personalización}
En este apartado se va a tener en cuenta la personalización que debe tener la aplicación teniendo en cuenta el tipo de usuario.

\begin{requisitostbl}{X[-1]X[1]X[1]X[1]X[1]}
    ID & Tipo & Categoría & Prioridad &  Dependencias \\
    1  & Funcional & Personalización & Must &   \\

    Idioma de la aplicación  \\

    \textbf{Descripción}:

    La aplicación debe estar traducida en los siguientes idiomas:
    \begin{itemize}
        \item Castellano
        \item Inglés
        \item Francés
        \item Italiano
        \item Alemán
        \item Portugués
    \end{itemize}
    \\

    \textbf{Razón}:

    La aplicación está dirigida a usuarios de toda Europa y América.  \\
\end{requisitostbl}

\begin{requisitostbl}{X[-1]X[1]X[1]X[1]X[1]}
    ID & Tipo & Categoría & Prioridad &  Dependencias \\
    1  & Funcional & Personalización & Must &   \\

    Cambio automático del idioma en la aplicación  \\

    \textbf{Descripción}:

    La aplicación debe tener en cuenta el idioma del navegador y utilizarlo como idioma principal de la aplicación.
    \\

    \textbf{Razón}:

    Lo habitual es que el idioma predefinido del navegador sea el principal del usuario  \\
\end{requisitostbl}


\subsection{Requisitos no funcionales}
Siguiendo con el análisis de las entrevistas realizadas con la \textbf{AIC} se han determinado la existencia de requisitos no funcionales que también deben ser cumplidos en el proyecto.

Una vez más, \textcite{Sommerville2005} nos indica que los requisitos no funcionales son aquellos que no se refieren directamente a las funciones específicas que proporciona el sistema.

Estos requisitos no funcionales también los podemos diferenciar en distintos apartados como vamos a hacer a continuación.

\subsubsection{Requisitos de producto}
Estos son los requerimientos que especifican el comportamiento del producto y que pueden verse asociados a la eficiencia, fiabilidad, disponibilidad...

A continuación se detallan todos los requisitos:

\begin{requisitostbl}{X[-1]X[1]X[1]X[1]X[1]}
    ID & Tipo & Categoría & Prioridad &  Dependencias \\
    1  & No Funcional & Producto & Must &   \\

    Disponibiliad de la aplicación \\

    \textbf{Descripción}:

    La aplicación web debe estar disponible 24x7 los 365 días del año.  \\

    \textbf{Razón}:

    Dado que es una aplicación que va a dar cobertura a Europa y América, debe estar online en todas las zonas horarias.  \\
\end{requisitostbl}

\begin{requisitostbl}{X[-1]X[1]X[1]X[1]X[1]}
    ID & Tipo & Categoría & Prioridad &  Dependencias \\
    2  & No Funcional & Producto & Must & 1  \\

    Despliegue de la aplicación \\

    \textbf{Descripción}:

    Se hará uso del sistema “blue-green” a la hora de realizar el despliegue de nuevas versiones de la aplicación.  \\

    \textbf{Razón}:

    Los sistemas de despliegue “blue-green” evitan tener que poner la web en mantenimiento durante actualizaciones. \\

    \textbf{Referencias}:

    Para más información acerca de los despliegues “blue-green” visitar la siguiente \href{https://www.redhat.com/en/topics/devops/what-is-blue-green-deployment}{url}.
\end{requisitostbl}

\begin{requisitostbl}{X[-1] X X X X}
    ID & Tipo & Categoría & Prioridad &  Dependencias \\
    3  & No Funcional & Producto & Must &   \\
    Escalabilidad de la aplicación  \\

    \textbf{Descripción}:

    La aplicación debe escalar de manera automática cuando la carga media de las máquinas virtuales supere el 70\% \\

    \textbf{Razón}:

    Para evitar que la carga de las máquinas llegue a impedir el acceso a la aplicación por parte de los usuarios, necesitamos un sistema auto-escalable.
    \\
\end{requisitostbl}

\begin{requisitostbl}{X[-1] X X X X}
    ID & Tipo & Categoría & Prioridad &  Dependencias \\
    4  & No Funcional & Producto & Must &  5 \\
    Portabilidad de la aplicación \\

    \textbf{Descripción}:

    La aplicación debe ser portable entre plataformas, es por ello que se va a crear en formato \textit{Progressive Web App} (\textbf{PWA}).
    \\

    \textbf{Razón}:

    Dado que es una aplicación web va a ser utilizada por distintos usuarios y desde distintas plataformas, debe verse de manera correcta en todos los dispositivos \\
\end{requisitostbl}


\subsubsection{Requisitos organizacionales}

\begin{requisitostbl}{X[-1] X X X X}
    ID & Tipo & Categoría & Prioridad &  Dependencias \\
    5  & No Funcional & Producto & Must &  4 \\

    Uso de estándares \\

    \textbf{Descripción}:

    Se hará uso de los estándares web necesarios (HTML5, CSS, todas las especificaciones W3C) para crear la aplicación web.
    \\

    \textbf{Razón}:

    Dado que vamos a crear una \textit{Progressive Web App}, es necesario cumplir los estándares web. \\
\end{requisitostbl}


%\subsubsection{Requisitos externos}
%En este
%
%\subsubsection{Escalabilidad}
%
%\subsubsection{Funcionalidad}
%
%\subsubsection{Usabilidad}
%
%\subsubsection{Accesibilidad}
%
%\subsubsection{...}

\section{Especificación}

\chapter{Modelo conceptual}

\chapter{Modelado de datos}

\chapter{Modelado de interfaces}

\chapter{Diseño}


\chapter{Conclusiones}


\printbibliography[title={Referencias bibliográficas},heading=bibintoc]

\end{document}
