\newcommand{\ClassPath}{../VIU_TFM_LaTeX_template}
\documentclass{\ClassPath/viu-tfm-template}


\definecolor{maincolor}{HTML}{e65218}

%--------------------------------------------------------------------------
% Definiciones necesarias Modifica con tus datos
%--------------------------------------------------------------------------
\def\nombre{Gómez Olivencia, Rubén}
\def\dni{78910013-A}
\def\titulo{Análisis y diseño de la aplicación web \linebreak\linebreak\linebreak para la Asociación Internacional de Chefs}
\def\titulacion{Máster Universitario en Desarrollo de Aplicaciones y Servicios Web}
\def\curso{2022-2023}

%Los siguientes son opcionales: si no se ponen, la portada cambia un poco. Ideal para escribir artículos/trabajos cortos
\def\dirige{}
\def\convocatoria{}
\def\asignatura{Ingeniería Software Web}


% importar fichero de Bibliografía
\addbibresource{Actividad_2.bib}

\begin{document}
    \graphicspath{{../VIU_TFM_LaTeX_template/}}

    \coverpage

    \tableofcontents

\chapter{Introducción}
En el siguiente documento se va a detallar el proceso que nos ha llevado para la realización de la aplicación web solicitada por la  \textbf{Asociación Internacional de Chefs (AIC)} para la gestión y compartición de recetas culinarias.

A lo largo del documento se diferenciarán distintos apartados en los que se detallarán cuál ha sido la metodología empleada durante todo el proceso, la planificación realizada y cómo se ha hecho la obtención y posterior análisis de requisitos.

Con todo ello, se ha realizado el modelado de datos que es necesario para llevar a cabo la petición del cliente, así como distintos diseños conceptuales que serán explicados en sus respectivos apartados.

Para finalizar se han incluido las conclusiones de todo el proceso llevado a cabo.


\chapter{Metodología utilizada}
Para llevar a cabo la creación de la aplicación web, y todo el proceso subyacente, se ha hecho uso del método “diseño centrado en el usuario” (\textbf{DCU}), y tal como nos dice la definición de \textcite{hassan2004diseno} “se caracteriza por asumir que todo el proceso de diseño y desarrollo del sitio web debe estar conducido por el usuario, sus necesidades, características y objetivos”.

Es por eso, que para llevar a buen puerto el proyecto encargado por la AIC, debemos de comprender las funcionalidades presentadas y sus requerimientos buscando como objetivo final el tener la mejor experiencia de usuario.

\chapter{Planificación}

\chapter{Requisitos del cliente}

En este apartado vamos a detallar cómo se ha realizado toda la toma de requisitos y el análisis para poder realizar una diferenciación de los mismos para su posterior modelado, tanto de datos como de interfaces.

Antes de continuar, cabe recordar qué es un requisito, y tal como nos dice \textcite{Sommerville2005}, es una declaración abstracta de alto nivel de un servicio que debe proporcionar el sistema o una restricción de éste.

Así mismo, y teniendo en cuenta la definición proporcionada por la \textcite{IEEE610} en su estándar 610.12-1990, se define como “una condición o capacidad que debe estar presente en un sistema o componentes de sistema para satisfacer un contrato, estándar, especificación u otro documento formal”.


\section{Obtención de los requisitos}
Para la obtención de los requisitos se realizaron una serie de entrevistas con los responsables de la \textbf{AIC} con el fin de recabar toda la información posible para que de esta manera quedasen claro las funcionalidades y requisitos mínimos que debe tener la aplicación web.

A la reunión acudieron, como ya se ha dicho, los responsables de negocio de la Asociación Internacional de Chefs (\textbf{AIC}) y por otro lado, y en función de analista, Rubén Gómez Olivencia.




\section{Análisis de requisitos}



\section{transaccionales}


\section{Requisitos funcionales}

\begin{enumerate}[label=\alph*)]
    \item Registrar el usuario a la plataforma de la AIC.
    \item Autenticar usuario al ingresar a la aplicación de la AIC.
    \item Cargar la receta, con todos sus aspectos asociados (categoría, lista de
    ingredientes, procedimiento, dificultad, costo, etc.)
    \item Buscar recetas en base a distintos aspectos: clase de receta, tipo de ingrediente
    de base, nivel de dificultad, costo de la receta.
    \item Visualizar resultados de búsqueda.
    \item Visualizar la información sobre cada receta.
    \item Compartir receta en las redes.
    \item Valorar receta (de 1 a 5 estrellas).
\end{enumerate}

\section{Requisitos no funcionales}

\subsection{Disponibilidad}

Meter cosas de: varios idiomas,

\subsection{Escalabilidad}

\subsection{Funcionalidad}

\subsection{Usabilidad}

\subsection{Accesibilidad}

\subsection{...}

\section{Especificación}

\chapter{Modelo conceptual}

\chapter{Modelado de datos}

\chapter{Modelado de interfaces}

\chapter{Diseño}


\chapter{Conclusiones}


\printbibliography[title={Referencias bibliográficas},heading=bibintoc]

\end{document}
