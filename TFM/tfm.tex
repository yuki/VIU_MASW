\newcommand{\ClassPath}{../VIU_TFM_LaTeX_template}
\documentclass{\ClassPath/viu-tfm-template}
\usepackage{multicol}

\definecolor{maincolor}{HTML}{f25416}

%--------------------------------------------------------------------------
% Definiciones necesarias Modifica con tus datos
%--------------------------------------------------------------------------
\def\nombre{ Gómez Olivencia, Rubén}
\def\dni{78910013-A}
\def\titulo{Gestor de contraseñas y documentación \linebreak\linebreak sensible en entorno multiusuario \linebreak\linebreak hecho con Angular y Hashicorp Vault}
\def\subtitulo{(Trabajo Fin de Máster)}
\def\titulacion{Máster Universitario en Desarrollo de Aplicaciones y Servicios Web}
\def\curso{2022-2023 (Ed. Abril)}

%Los siguientes son opcionales: si no se ponen, la portada cambia un poco. Ideal para escribir artículos/trabajos cortos
\def\dirige{Simarro Moncholí, Héctor}
\def\codirige{de Fez Lava, Ismael}
\def\convocatoria{Primera}
\def\asignatura{}


% importar fichero de Bibliografía
\addbibresource{tfm.bib}

\begin{document}
\graphicspath{{../VIU_TFM_LaTeX_template/}}

\coverpage


%--------------------------------------------------------------------------
% Abstract
%--------------------------------------------------------------------------

% Creo un “abstract” propio, porque la plantilla “book” no la tiene, y cambiar a “report” no aporta nada nuevo. Aparte, el comando “abstract” original hace salto de página.

\vspace*{\fill}
\begin{center}
    \textbf{Resumen}
\end{center}

La gestión de contraseñas y documentación que una empresa acumula puede llegar a resultar un problema si no se sabe gestionar bien. Esta documentación puede crecer si es una empresa de servicios informáticos, con varios clientes, distintos proyectos, contraseñas para distintos servicios...

Existen distintas herramientas que están especializadas en gestión de contraseñas, o gestión de documentación, pero quizá no de ambos, lo que requeriría tener dos aplicaciones distintas, y muchas veces es conveniente tener la documentación de lo realizado junto con la contraseña de acceso. Aún con todo, pueden no ajustarse a nuestras necesidades, y salvo que sea una aplicación de Software Libre, no podremos realizar modificaciones.

Aparte, estas aplicaciones pueden ser de pago, no sabemos la seguridad real de los datos, normalmente la información no está en nuestros servidores y pueden existir fallos de seguridad que expondrán nuestros datos.

Este TFM trata de aproximar la creación de un gestor de contraseñas y documentación sensible creado en Angular y haciendo uso de Hashicorp Vault como almacenamiento seguro.

\keywords{gestión de contraseñas, seguridad, autenticación, autorización, documentación sensible, cifrado de datos, hashicorp vault}

\vspace*{\fill}
\vspace*{\fill}
\vspace*{\fill}

\pagebreak

%--------------------------------------------------------------------------
% end of Abstract
%--------------------------------------------------------------------------


\tableofcontents


\chapter{Introducción}
En la era digital en la que nos encontramos guardar datos e información es relativamente sencillo, gracias a que cada vez conseguimos tener discos duros con mayor capacidad. En caso de que un único disco duro se nos quede pequeño, existen sistemas como RAID que nos permite combinar varios discos añadiendo tolerancia a fallos en caso de rotura de alguno de ellos.

Por otro lado, tenemos información sensible como son las contraseñas, que a pesar de ocupar poco espacio, debemos tener especial cuidado con ellas, ya que no deben ser accedidas ni utilizadas por personas que no sean su responsabilidad.

En el ámbito personal existen distintas posibilidades que nos permiten gestionar las contraseñas, ya sea de manera integrada en los navegadores web, o a través de aplicaciones externas.

Cuando esta gestión de contraseñas se pasa al ámbito empresarial, a pesar de que también existen distintas alternativas, puede resultar más complicado el utilizarlas. Pueden no contar con características necesarias para la empresa, quizá la integración con los sistemas de la empresa no sea posible, si es una opción de pago, sea demasiado cara...

A las contraseñas, dentro de una empresa también hay que añadir la información sensible que maneja, ya sea propia o de terceras empresas (clientes o proveedores que se tengan). Esa información, ya no tiene por qué ceñirse a “usuario” y “contraseña”, si no que puede ser documentación, ficheros, imágenes, ...

Encontrar una aplicación que se ciña a todas las necesidades actuales puede resultar complejo, y más complejo si estas necesidades varían a lo largo del tiempo,. La aplicación que utilicemos puede ser abandonada por los creadores; si es de pago, la empresa puede cerrar; la posibilidad de añadir características propias no existirá salvo que sea Software Libre, ... puede haber muchos inconvenientes.

A lo largo de este \textbf{trabajo final del Máster Universitario en Desarrollo de Aplicaciones y Servicios Web} se analizará la creación de un sistema gestor de contraseñas y documentación sensible para entornos multiusuarios hecho con el \textit{frontend} de desarrollo \href{https://angular.io/}{Angular} y utilizando  \href{https://www.vaultproject.io/}{Hashicorp Vault} como sistema de backend.


\chapter{Objetivos}
%TODO: añadir introducción


\begin{itemize}
    \item El gestor de contraseñas debe contar con un \textbf{sistema de autenticación}. Sólo debe permitir el acceso a las personas que hayan pasado algún sistema de verificación que demuestre ser quién es.

    \item También debe existir un \textbf{sistema de autorización, que controle el tipo de acceso y permisos} que tenga en en cuenta quién los realiza y a qué contraseñas se quiere acceder.

    \item Para tener un registro de lo sucedido, también es importante contar con un \textbf{sistema de auditoría}. A través de él se registrarán los accesos y operaciones que se realizan.

    \item Lógicamente, la \textbf{seguridad en el acceso y en el almacenamiento} debe ser la principal prioridad. Hay que evitar que cualquier tipo de acceso a los datos, ya sea a través del almacenamiento, o por la transmisión de los datos, no esté cifrado. De esta manera, la información interceptada, al estar cifrada, carecerá de sentido.
\end{itemize}

La gestión de contraseñas



\chapter{Estado del arte}

\section{Gestores de contraseñas conocidos}


\chapter{Metodología}


\chapter{Resultados}


\chapter{Conclusiones}


\chapter{Trabajo futuro}


\end{document}
