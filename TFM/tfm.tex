\newcommand{\ClassPath}{../VIU_TFM_LaTeX_template}
\documentclass{\ClassPath/viu-tfm-template}
\usepackage{multicol}

\definecolor{maincolor}{HTML}{f25416}

%--------------------------------------------------------------------------
% Definiciones necesarias Modifica con tus datos
%--------------------------------------------------------------------------
\def\nombre{ Gómez Olivencia, Rubén}
\def\dni{78910013-A}
\def\titulo{Gestor de contraseñas y documentación \linebreak\linebreak sensible en entorno multiusuario \linebreak\linebreak hecho con Angular y Hashicorp Vault}
\def\subtitulo{(Trabajo Fin de Máster)}
\def\titulacion{Máster Universitario en Desarrollo de Aplicaciones y Servicios Web}
\def\curso{2022-2023 (Ed. Abril)}

%Los siguientes son opcionales: si no se ponen, la portada cambia un poco. Ideal para escribir artículos/trabajos cortos
\def\dirige{Simarro Moncholí, Héctor}
\def\codirige{de Fez Lava, Ismael}
\def\convocatoria{Primera}
\def\asignatura{}


% importar fichero de Bibliografía
\addbibresource{tfm.bib}

\begin{document}
\graphicspath{{../VIU_TFM_LaTeX_template/}}

\coverpage


%--------------------------------------------------------------------------
% Abstract
%--------------------------------------------------------------------------

% Creo un “abstract” propio, porque la plantilla “book” no la tiene, y cambiar a “report” no aporta nada nuevo. Aparte, el comando “abstract” original hace salto de página.

\vspace*{\fill}
\begin{center}
    \textbf{Resumen}
\end{center}

La gestión de contraseñas y documentación que una empresa acumula puede llegar a resultar un problema si no se sabe gestionar de manera correcta. Esta documentación puede crecer si es una empresa de servicios informáticos, con varios clientes, distintos proyectos, contraseñas para distintos servicios...

Existen distintas herramientas que están especializadas en gestión de contraseñas y otras en gestión de documentación, pero esto requeriría tener dos aplicaciones distintas. Muchas veces es conveniente tener la documentación de lo realizado junto con la contraseña de acceso, para así poder facilitar la administración de servicios.

Aparte, estas aplicaciones pueden ser de pago, no tendremos la certeza de cómo es la seguridad real de los datos, ya que normalmente la información no está en nuestros servidores y pueden existir fallos de seguridad que expongan nuestra información. También hay que tener en cuenta que pueden no ajustarse a nuestras necesidades y salvo que sea una aplicación de Software Libre, no podremos realizar modificaciones

Este TFM trata de aproximar la creación de un gestor de contraseñas y documentación sensible creado en Angular y haciendo uso de Hashicorp Vault como almacenamiento seguro.

\keywords{gestión de contraseñas, seguridad, autenticación, autorización, documentación sensible, cifrado de datos, hashicorp vault}

\vspace*{\fill}
\vspace*{\fill}
\vspace*{\fill}

\pagebreak

%--------------------------------------------------------------------------
% end of Abstract
%--------------------------------------------------------------------------


\tableofcontents

\chapter{Introducción}
En la era digital en la que nos encontramos guardar datos e información es relativamente sencillo, gracias a que cada vez conseguimos tener discos duros con mayor capacidad. En caso de que un único disco duro se nos quede pequeño, existen sistemas como RAID que nos permite combinar varios discos añadiendo tolerancia a fallos en caso de rotura de alguno de ellos.

Por otro lado, tenemos información sensible como son las contraseñas, que a pesar de ocupar poco espacio, debemos tener especial cuidado con ellas, ya que no deben ser accedidas ni utilizadas por personas que no estén autorizadas para ello.

En el ámbito personal existen distintas aplicaciones que nos permiten gestionar contraseñas, ya sea de manera integrada en los navegadores web, o a través de aplicaciones externas.

Cuando esta gestión de contraseñas se pasa al ámbito empresarial, a pesar de que también existen distintas alternativas, puede que no cuenten con características necesarias para la empresa; quizá la integración con los sistemas de la empresa no sea posible; en caso de ser una opción de pago, ser demasiado cara por el volumen de empleados que tenemos... Puede resultar complejo decantarse por una solución existente.

A las contraseñas también hay que añadir la información sensible que maneja la empresa, ya sea información propia o de terceras empresas (clientes o proveedores que se tengan). Esa información, ya no tiene por qué ceñirse a “usuario” y “contraseña”, si no que puede ser documentación, ficheros, imágenes, esquemas de red ...

Encontrar una aplicación que se ciña a todas las necesidades puede resultar complejo, y más si estas necesidades varían a lo largo del tiempo. También pueden surgir distintos problemas: la aplicación que utilicemos puede ser abandonada por los creadores; si es de pago, el precio o la suscripción puede variar; la posibilidad de añadir características propias no será posible salvo que sea Software Libre, ...

A lo largo de este \textbf{trabajo final del Máster Universitario en Desarrollo de Aplicaciones y Servicios Web} se analizará la creación de un sistema gestor de contraseñas y documentación sensible para entornos multiusuario, hecho con el \textit{frontend} de desarrollo \href{https://angular.io/}{Angular} y utilizando  \href{https://www.vaultproject.io/}{Hashicorp Vault} como sistema de backend.

\section{Motivación del proyecto}

La motivación principal del proyecto es la de crear una aplicación que gestione contraseñas y documentación tratando de buscar un enfoque generalista para que pueda ser utilizada en empresas tecnológicas.

Algunas de las aplicaciones de gestión de contraseñas más conocidas están enfocadas sólo al almacenamiento de contraseñas, y aunque permitan subir ficheros para ser cifrados, al querer realizar modificaciones, hay que descargar el fichero, realizar las modificaciones y posteriormente volver a subirlo.

Estas aplicaciones, de las cuales más adelante se analizarán algunas, la versión multi-usuario suele ser de pago, y las que lo tienen, no son ni Software Libre ni permiten la posibilidad de tener el servicio \textit{on premise}.



\chapter{Objetivos}

El objetivo principal es crear una aplicación que gestione contraseñas y en la que poder escribir y guardar documentación sensible, basada en un entorno web, que pueda ser utilizada en un entorno multiusuario como una empresa.

Este gestor de contraseñas deberá cumplir con unos requisitos mínimos para que pueda ser desplegado y utilizado en una empresa para mejorar la seguridad de los datos que se guardan.


\section{Requisitos del proyecto}
Como propuesta para la creación de un gestor de contraseñas propio, se han identificado los siguientes requisitos que deben de cumplirse para considerar que el trabajo final de máster ha cumplido con las funcionalidades iniciales citadas previamente:

\begin{itemize}

    \item Se quiere crear un sistema que \textbf{gestione contraseñas e información sensible}. Esta información sensible puede ser imágenes, documentos, o ficheros en general que la aplicación permitirá subir.

    También se podrá generar documentación en la propia aplicación a través de un interfaz \textbf{WYSIWYG} (\textit{what you see is what you get}) que guarde la información en formato Markdown.

    \item El sistema estará basado en una aplicación web creada con el \textit{framework frontend} \href{https://angular.io/}{Angular} tratando que sea lo más modular posible.

    La modularidad nos va a permitir poder añadir características nuevas a la aplicación en el futuro. De esta manera, podremos adaptar la aplicación a posibles exigencias futuras que necesite la empresa.

    \item El gestor de contraseñas debe contar con un \textbf{sistema de autenticación}. Sólo debe permitir el acceso a las personas que hayan pasado algún sistema de verificación que demuestre ser quién es.

    Este sistema de autenticación no tiene por qué limitarse al clásico “usuario y contraseña”, ya que podrá adaptarse para en el futuro utilizar, de forma sencilla, otros sistemas conocidos, como por ejemplo: certificados \textbf{TLS}, contraseñas de un sólo uso basadas en tiempo (TOTP), tokens basados en \href{https://en.wikipedia.org/wiki/JSON_Web_Token}{JWT}, sistemas \href{https://en.wikipedia.org/wiki/OAuth}{OAuth}/\href{https://en.wikipedia.org/wiki/OpenID#OpenID_Connect_(OIDC)}{OIDC}, ...

    \item Debe existir un \textbf{sistema de autorización, que controle el tipo de acceso y permisos} que tenga en cuenta quién los realiza y a qué contraseñas se quiere acceder.

    Estos permisos pueden basarse en \textbf{roles}, para de esta manera poder agrupar usuarios, y en base a ellos facilitar la autorización

    \item Para tener un registro de lo sucedido, también es importante contar con un \textbf{sistema de auditoría}. A través de él se registrarán los accesos y operaciones que se realizan.

    \item Lógicamente, la \textbf{seguridad en la transmisión y en el almacenamiento} debe ser la principal prioridad.  \textcite{scarfone2009guide} nos recuerdan que las comunicaciones que contienen contraseñas deben ser cifradas, ya sea mediante uso del protocolo TLS (\textit{Transport Layer Security}) o creando un túnel en la comunicación a través de una \textbf{VPN}.

    En caso de que algún atacante obtuviese acceso físico al sistema de almacenamiento, de nuevo, los datos estarían cifrados, evitando la fuga de información.

    \item La aplicación debe ser \textbf{sencilla de utilizar}, para que al usuario final no le cueste utilizarla. De esta manera \textbf{conseguiremos que el usuario final la integre en su metodología de trabajo} a la hora de crear/usar contraseñas y guardar información sensible.

    \item Es importante que la aplicación \textbf{se adapte a la pantalla} donde se esté visualizando. De esta manera el usuario podrá hacer uso de ella a través de dispositivos móviles.

    \item Para el almacenamiento de la información y el sistema de autorización se hará uso del sistema \href{https://www.vaultproject.io/}{Vault} de la empresa \href{https://www.hashicorp.com/}{Hashicorp}.
\end{itemize}

Teniendo en cuenta las metodologías utilizadas, y explicadas en apartados posteriores, cualquier añadido a estos requisitos se podría añadir en etapas posteriores del desarrollo.


\chapter{Marco tecnológico}

\textcite{tapas} nos indican que la categoría de software donde podemos meter a los gestores de contraseñas es amplia y pueden contener muchas técnicas diferentes, pero que generalmente pueden ser complementarias. Es por ello que se van a identificar algunas de las características que suelen proveer este tipo de software:

\begin{itemize}
    \item Almacenar y devolver contraseñas.
    \item Detectar la fuerza de las contraseñas.
    \item Convertir otros tipos de autenticación a contraseñas.
    \item Detección de formularios en webs para autorrellenado de datos.
    \item Aplicaciones con almacenamiento en la nube y sincronización entre dispositivos.
\end{itemize}

Este tipo de aplicaciones suelen hacer uso de una contraseña maestra para proteger el “almacén de contraseñas”, pero esto puede desencadenar en dos inconvenientes:

\begin{itemize}
    \item En el estudio de \textcite{belenko2012secure} analizan más de una docena de gestores de contraseñas y concluyen que muchos de ellos no proveen un nivel de seguridad apropiado.

    \item La contraseña elegida por el usuario puede no ser resistente a un ataque si el “almacén de contraseñas” es robado.
\end{itemize}

Es cierto que en los últimos años, cuando este tipo de software es utilizado en dispositivos móviles la contraseña maestra se puede sustituir por un reconocimiento biométrico (como puede ser reconocimiento de la huella dactilar o reconocimiento facial). En cambio, cuando se refiere a aplicaciones de escritorio, esto no sucede.


\section{Gestores de contraseñas conocidos}

La siguiente tabla muestra una comparativa con distintos gestores de contraseñas conocidos y distintas característica que pueden ser útiles dentro de una empresa:


\begin{yukitblrcol}{X[9]X[7]X[7]X[8]X[7]}
    &    \titlehref{https://keepass.info/}{KeePass} &
         \titlehref{https://keeweb.info/}{KeeWeb} &
         \titlehref{https://1password.com/}{1Password} &
         \titlehref{https://www.lastpass.com/}{LastPass}
         \\
    Multiusuario  & {\LARGE \xmark}  & {\LARGE \cmark} & {\LARGE \xmark} / {\LARGE \cmark}** & {\LARGE \xmark} / {\LARGE \cmark}** \\
    Distintos sistemas de autenticación \textbf{*}  & {\LARGE \xmark}  &  {\LARGE \xmark} & {\LARGE \xmark} / {\LARGE \cmark}*** &  {\LARGE \xmark} / {\LARGE \cmark}*** \\
    Permite guardar ficheros & {\LARGE \cmark} & {\LARGE \cmark} & {\LARGE \cmark} & {\LARGE \cmark}\\
    Permite editar documentación & {\LARGE \xmark} & {\LARGE \xmark} & {\LARGE \xmark} & {\LARGE \xmark} \\
    Versión web  & {\LARGE \xmark}  & {\LARGE \cmark} & {\LARGE \cmark} & {\LARGE \cmark} \\
    Versión escritorio  &  {\LARGE \cmark}  & {\LARGE \cmark} & {\LARGE \cmark} & {\LARGE \cmark} \\
    App móvil  & {\LARGE \xmark} / {\LARGE \cmark}**** & {\LARGE \xmark} & {\LARGE \cmark} & {\LARGE \cmark} \\
    Informes de auditoría &  {\LARGE \xmark}***** & {\LARGE \xmark} & {\LARGE \xmark} / {\LARGE \cmark}** & {\LARGE \xmark} / {\LARGE \cmark}**  \\
    On-premises  & {\LARGE \cmark}  & {\LARGE \cmark} & {\LARGE \xmark} & {\LARGE \xmark}\\
    Coste  &  0€  & 0€ &  \$7,99\linebreak user/mes** & 5,70€\linebreak user/mes** \\
    Software Libre  & {\LARGE \cmark}  & {\LARGE \cmark} & {\LARGE \xmark} & {\LARGE \xmark}\\

\end{yukitblrcol}

\begin{itemize}
    \item[*] Permitir distintos sistemas de autorización como usuario+contraseña, autenticación contra Active Directory, sistema LDAP, TOTP...
    \item[**] La versión para empresas (“business” en 1password).
    \item[***] La versión para empresas y sólo sistemas “\textit{single sign on}”.
    \item[****] Versión no oficial.
    \item[*****] Se puede conseguir algo similar mediante un “\textit{trigger}”.
\end{itemize}

Tal como se ve en la tabla anterior, sólo las aplicaciones que son Software Libre tienen versión \textit{on-premises}, es decir, que podemos ejecutar y guardar los datos en nuestros servidores.


\section{Problemas acontecidos}

Dada la importancia de los datos que se puede guardar en un gestor de contraseñas y documentación, es importante que los datos se encuentren a buen recaudo, ya que cualquier fallo de seguridad puede suponer el robo de dicha información.

Es por eso que los gestores de contraseñas más conocidos han sufrido distintos ataques o se les han descubierto ciertas vulnerabilidades en las que ha habido filtrado y/o pérdida de información.

Algunos ejemplos en los últimos años:

\begin{itemize}
    \item \textcite{norton} informa que las cuentas de los usuarios que utilizan el programa \href{https://us.norton.com/feature/password-manager}{Norton Password Manager} han sido potencialmente comprometidas, obteniendo el usuario y contraseña, y habiendo obtenido nombre, apellidos, número de teléfono y la cuenta de mail de los usuarios.

    La empresa ha declarado a \textcite{norton2} que aunque los servidores no han sido comprometidos, es posible que se hayan usado los usuarios y contraseñas para acceder a las cuentas de usuario. Se estiman que el ataque se ha realizado sobre al menos 8.000 cuentas.

    \item La aplicación \href{https://www.lastpass.com/}{LastPass}, analizada previamente, ha sufrido varios ataques.

    En \textcite{lastpass}, la empresa declara que están trabajando “para comprender el alcance del incidente e identificar a qué información específica se ha accedido”. Por otro lado, \textcite{lastpass2} analiz, que aunque la empresa haya indicado que los datos cifrados obtenidos no se pueden descifrar, no cree en la palabra de la empresa.

    Para poder ver un breve resumen de los últimos ataques, \textcite{lastpass3} nos muestra los últimos cuatro ataques en tres años, con enlaces a cada noticia.

    \item Existen varios listados de gestores de contraseñas que han sufrido vulnerabilidades, tal como nos indica \textcite{hacked2}.

    En \textcite{hacked} se muestra un listado por año (desde el 2014) con varios gestores de contraseñas que han sufrido vulnerabilidades: 1password, RoboForm, Keepass, My1Login, ...

\end{itemize}

Como se puede ver, y este listado sólo es una pequeña muestra, los gestores de contraseña son un objetivo claro de ataques.  \textcite{hacked3} analiza parte de estos ataques y da pequeñas recomendaciones sobre cómo mejorar la protección (como modificar la contraseña por una más fuerte).


\section{Herramientas tecnológicas utilizadas}

Para la creación del proyecto se ha hecho uso de distintas herramientas tecnológicas, que a su vez también han ayudado con las metodologías ágiles que se expondrán más adelante.


\subsection{Framework Angular}

\href{https://angular.io/}{Angular} es un \textit{framework} que nos facilita la creación de aplicaciones web, utiliza el lenguaje de programación \href{https://www.typescriptlang.org/}{Typescript} y es Software Libre.

Angular permite programar haciendo uso de la arquitectura conocida como “\textbf{Modelo–vista–controlador}”, que  va a permitir separar la representación visual de la aplicación, el modelo de datos que se va a tener y la lógica de negocio.

Gracias a ello, la aplicación desarrollada va a ser lo más modular posible, para que de esta manera el mantenimiento y el añadirle nuevas características en el futuro se pueda realizar de manera sencilla.

Por otro lado, y gracias a que Angular es un \textit{framework} de lado de cliente (el resultado se ejecuta en el navegador del usuario), va a permitir que la carga se reduzca en el lado del servidor.

\subsection{Hashicorp Vault}

\href{https://www.vaultproject.io/}{Vault} es el proyecto de código abierto para asegurar, almacenar y controlar el acceso a secretos y datos sensibles creado por la empresa \href{https://www.hashicorp.com/}{Hashicorp} (conocidos también por la creación de \href{https://www.terraform.io/}{Terraform}).

Se va a utilizar Vault como sistema centralizado para varios aspectos de la aplicación web que se va a desarrollar:

\begin{itemize}
    \item \textbf{Gestión de autenticación}: Vault nos va a permitir tener un sistema de gestión de autenticación centralizado, que nos puede permitir pasar de un sistema de “usuario y contraseña”, a un sistema autenticado contra el Active Directory de la empresa, o hacer uso de sistemas sistemas \href{https://en.wikipedia.org/wiki/OAuth}{OAuth}/\href{https://en.wikipedia.org/wiki/OpenID#OpenID_Connect_(OIDC)}{OIDC}.

    \item \textbf{Gestión de autorización}: A la hora de permitir el acceso y permisos a los secretos Vault cuenta con un sistema interno que nos ayudará a realizar dicha gestión.

    \item \textbf{Gestor de almacenamiento}: Se puede hacer uso de distintos sistemas de \textit{backend} de almacenamiento: desde ficheros cifrados a distintas bases de datos (MySQL, PostgreSQL, Cassandra,...).

    Dependiendo del almacenamiento elegido, el crear un sistema en Alta Disponibilidad será más sencillo que en otros. De todas maneras, la posibilidad de migrar de un sistema de almacenamiento a otro siempre es posible

    \item \textbf{Alta Disponibilidad (HA)}: Teniendo en cuenta que un gestor de contraseñas y documentación sensible es una pieza fundamental en una empresa, es vital que exista la posibilidad de crear un sistema en Alta Disponibilidad.

    Vault permite crear, dependiendo de las necesidades, distintos tipos de sistemas en Alta Disponibilidad multi-nodo, multi-región, con distintos tipos de sistemas de  ...

\end{itemize}

Aunque Vault no está pensado inicialmente para guardar documentación, uno de sus sistemas de almacenamiento es “clave-valor”.  Este ha sido el sistema elegido para guardar las contraseñas, la documentación que se va a poder generar en la propia aplicación y los ficheros que se van a poder subir.


\subsection{Git y despliegues automáticos}

\href{https://es.wikipedia.org/wiki/Git}{Git} es el sistema de control de versiones creado por \href{https://es.wikipedia.org/wiki/Linus_Torvalds}{Linus Torvalds} para la gestión del código fuente de \href{https://es.wikipedia.org/wiki/N%C3%BAcleo_Linux}{Linux}. Hoy en día es el sistema más utilizado gracias a características tan importantes como la facilidad para crear ramas, su versatilidad para adaptarse al flujo de trabajo del desarrollador, integración con los entornos de desarrollo más importantes...

Gracias a los sistemas CI/CD (del inglés “\textit{Continuous Integration/Continuous Delivery}) integrados en las plataformas de repositorios más conocidas (\href{https://github.com/}{GitHub} o \href{https://about.gitlab.com/}{GitLab}) o a través de herramientas propias (como \href{https://www.jenkins.io/}{Jenkins}), el realizar despliegues automatizados de las nuevas versiones permitirá que la aplicación pueda ser rápidamente utilizada por los usuarios finales.

%A pesar de que Git permite ser descentralizado, se ha utilizado la plataforma \href{https://github.com/}{Github} como plataforma centralizadora y de esta manera también así disponer de una copia de seguridad del código fuente.
%
%\begin{center}
%    \includegraphics[width=0.8\linewidth]{img/commits.png}
%    \captionof{figure}{Histórico de varios commits realizados}
%\end{center}
%
%Aparte, también se ha integrado con Trello (a través de los plugins conocidos como “Power-Ups”) para así poder asociar las tareas a los commits que las han cumplimentado. De esta manera conseguiremos una relación “tarea ↔ código” que podremos utilizar para analizar si ha habido algún error durante el desarrollo, o fases como la de \textit{testing}.


%\subsection{Creación de entornos con Docker}
%Como dice \textcite{mouat}, los contenedores son un concepto que lleva existiendo décadas en sistemas Unix (con el comando “chroot”), pero \href{https://es.wikipedia.org/wiki/Docker_(software)}{Docker} cogió la tecnología existente de contenedores Linux y la amplió de varias maneras, sobre todo gracias a su sistema de imágenes portable.
%
%Gracias a utilizar distintas imágenes Docker con distintos servicios, se ha podido realizar despliegues de manera sencilla para distintos entornos (desarrollo, test y producción).
%
%También ha sido especialmente útil al poder realizar el desarrollo en distintos ordenadores, sin perder el tiempo en realizar instalaciones de librerías y ejecutables, creando el entorno de desarrollo en segundos.




\chapter{Metodologías utilizadas}

Hoy en día es conveniente hacer uso de las \textbf{metodologías ágiles} cuando realizamos la gestión de un proyecto, ya que cuentan con una serie de ventajas que nos van a permitir dar una respuesta más rápida y flexible ante posibles cambios durante la vida de desarrollo del producto.

Ventajas que podemos destacar de las metodologías ágiles:

\begin{itemize}
    \item Poder entregar resultados funcionales al cliente cada poco tiempo y así obtener \textit{feedback} de lo realizado.
    \item Seremos capaces de dar una respuesta ágil y flexible ante posibles cambios que puedan surgir (cambios en los requisitos, ante bajas de personal asignado al proyecto, dificultades durante el desarrollo...).
    \item Tratar de evitar burocracia innecesaria, para centrarnos de esta manera en el producto y sus funcionalidades.
    \item Tratar de involucrar al cliente y colaborar con él, para de esta manera conseguir un mayor valor al producto.
\end{itemize}

\section{Gestión del proyecto}

Para comenzar con la gestión del proyecto debemos contar con los requisitos listados previamente y transformarlos en las conocidas como “\textbf{historias de usuario}”.

En ellas plasmaremos, en un lenguaje sencillo de entender, las funcionalidades que el producto debe tener, a las que asociaremos otros datos como son:

\begin{itemize}
    \item Un \textbf{identificador único} para poder diferenciarlas del resto.
    \item Una \textbf{estimación de tiempos} de lo que creemos que podemos tardar en realizarlo.
    \item La \textbf{prioridad} inicial para poder ordenarlas entre ellas.
    \item El \textbf{tema principal} al que se relaciona la historia de usuario.
    \item Se añadirán posibles \textbf{dependencias} de otras historias de usuario.
    \item Una \textbf{descripción} de lo que se quiere conseguir.
    \item Unos \textbf{criterios de validación} por parte del cliente.
\end{itemize}

\subsection{Resultado del análisis realizado}

A continuación se van a plasmar las distintas “historias de usuario” obtenidas:

\begin{requisitostbl}{X[1]X[2]X[2]X[3]X[2]}
    ID & Estimación & Prioridad  & Tema &  Dependencias \\
    1  & 25 horas & 1  & Interfaz web &   \\

    Desarrollar “esqueleto” del interfaz web \\

    \textbf{Descripción}:
    Como usuario necesito un interfaz web para poder acceder a la aplicación y hacer uso de ella. Es la base del proyecto. \\

    \textbf{Criterios de validación}:
    El interfaz es \textit{\textbf{responsive}}, visualmente atractivo y funcional. \\
\end{requisitostbl}

\begin{requisitostbl}{X[1]X[2]X[2]X[3]X[2]}
    ID & Estimación & Prioridad  & Tema &  Dependencias \\
    2  & 10 horas & 1  & Gestión de usuarios & 1  \\

    Crear “login” de usuario \\

    \textbf{Descripción}:
    Como usuario quiero poder acceder a la aplicación utilizando como sistema de autenticación un usuario y contraseña.  \\

    \textbf{Criterios de validación}:
    Los valores introducidos serán autenticados contra el \textit{backend} que se decida (base de datos o LDAP/Active Directory inicialmente). \\
\end{requisitostbl}

\begin{requisitostbl}{X[1]X[2]X[2]X[3]X[2]}
    ID & Estimación & Prioridad  & Tema &  Dependencias \\
    3  & 10 horas & 1  & Gestión de Secretos & 2  \\

    Crear secretos \\

    \textbf{Descripción}:
    Como usuario quiero poder crear un secreto para poder guardar documentación sensible y/o contraseñas. \\

    \textbf{Criterios de validación}:
    \begin{itemize}
        \item El botón para crear siempre será visible.
        \item Se podrá elegir la ruta (por defecto será donde nos encontramos) y el nombre del secreto.
    \end{itemize}
    \\
\end{requisitostbl}



\begin{requisitostbl}{X[1]X[2]X[2]X[3]X[2]}
    ID & Estimación & Prioridad  & Tema &  Dependencias \\
    4  & 30 horas & 1  & Gestión de Secretos & 2, 3  \\

    Listado de secretos \\

    \textbf{Descripción}:
    Como usuario quiero poder obtener un listado de los secretos existentes para poder navegar por ellos y seleccionar el que me interese. \\

    \textbf{Criterios de validación}:
    \begin{itemize}
        \item El listado completo se visualiza mediante un árbol jerarquizado.
        \item También se puede navegar como si fuera un explorador de archivos.
        \item El interfaz mostrará un \textit{breadcrumb} para indicar la ruta en la que nos encontramos.
        \item Se puede buscar por el nombre de los secretos.
    \end{itemize}
    \\
\end{requisitostbl}


\begin{requisitostbl}{X[1]X[2]X[2]X[3]X[2]}
    ID & Estimación & Prioridad  & Tema &  Dependencias \\
    5  & 15 horas & 2  & Gestión de Secretos & 2, 3, 4  \\

    Visualizar un secreto \\

    \textbf{Descripción}:
    Como usuario quiero acceder a un secreto para visualizar su contenido.  \\

    \textbf{Criterios de validación}:
    \begin{itemize}
        \item La visualización mostrará el secreto en formato HTML.
        \item Si un usuario accede a un secreto “bloqueado” el sistema lo debe alertar y no se le permitirá editarlo.
        \item El interfaz nos mostrará la última vez que se modificó el secreto.
        \item Al visualizar el secreto aparecerán botones con distintas acciones que se podrán realizar sobre él (editar, imprimir, ver históricos, borrar).
    \end{itemize}
    \\
\end{requisitostbl}


\begin{requisitostbl}{X[1]X[2]X[2]X[3]X[2]}
    ID & Estimación & Prioridad  & Tema &  Dependencias \\
    6  & 35 horas & 2  & Gestión de Secretos & 2, 3  \\

    Modificar un secreto \\

    \textbf{Descripción}:
    Como usuario quiero poder editar un secreto para realizar modificaciones.  \\

    \textbf{Criterios de validación}:
    \begin{itemize}
        \item La modificación se realizará a través de un editor \textit{\textbf{WYSIWYG}}.
        \item El editor permitirá realizar modificaciones en formato \href{https://es.wikipedia.org/wiki/Markdown}{Markdown}.
        \item El editor permitirá guardar el secreto y al hacerlo volveremos a la visualización del mismo.
        \item Al comenzar la modificación, el secreto se pondrá en estado “bloqueado”.
    \end{itemize}
    \\
\end{requisitostbl}

\vspace{20pt}

\begin{requisitostbl}{X[1]X[2]X[2]X[3]X[2]}
    ID & Estimación & Prioridad  & Tema &  Dependencias \\
    7  & 10 horas & 3  & Gestión de Secretos & 2, 3, 5  \\

    Borrado de secretos \\

    \textbf{Descripción}:
    Como usuario quiero poder borrar secretos para que desaparezcan.  \\

    \textbf{Criterios de validación}:
    \begin{itemize}
        \item Antes de borrar debe existir un mensaje de confirmación.
        \item Al borrar un secreto, el árbol de secretos se debe actualizar.
    \end{itemize}
     \\
\end{requisitostbl}

\vspace{20pt}

\begin{requisitostbl}{X[1]X[2]X[2]X[3]X[2]}
    ID & Estimación & Prioridad  & Tema &  Dependencias \\
    8  & 15 horas & 3  & Gestión de Secretos & 2  \\

    Permitir gestionar ficheros como secretos \\

    \textbf{Descripción}:
    Como usuario quiero poder subir ficheros al sistema para guardarlos cifrados.  \\

    \textbf{Criterios de validación}:
    \begin{itemize}
        \item Se permite subir cualquier tipo de fichero.
        \item Se permite descargar el fichero original.
        \item Se debe poder visualizar los ficheros más habituales en la propia web (imágenes y ficheros PDF) sin necesidad de descargarlos.
    \end{itemize}
     \\
\end{requisitostbl}


\begin{requisitostbl}{X[1]X[2]X[2]X[3]X[2]}
    ID & Estimación & Prioridad  & Tema &  Dependencias \\
    9  & 20 horas & 4  & Gestión de Secretos & 3, 6  \\

    Versionado de secretos\\

    \textbf{Descripción}:
    Como usuario quiero poder ver las veces que el fichero ha sido modificado para poder ver los cambios realizados.  \\

    \textbf{Criterios de validación}:
    \begin{itemize}
        \item Poder ver el número de versiones que tiene un secreto.
        \item Poder ver una versión concreta del secreto.
        \item Poder ver las diferencias entre versiones.
        \item Poder restaurar una versión concreta del secreto.
    \end{itemize} \\
\end{requisitostbl}


\begin{requisitostbl}{X[1]X[2]X[2]X[3]X[2]}
    ID & Estimación & Prioridad  & Tema &  Dependencias \\
    10  & 5 horas & 4  & Gestión de Secretos & 2, 5  \\

    Imprimir secretos \\

    \textbf{Descripción}:
    Como usuario quiero poder imprimir un secreto. \\

    \textbf{Criterios de validación}:
    Existe un botón para imprimir un secreto. \\
\end{requisitostbl}


\begin{requisitostbl}{X[1]X[2]X[2]X[3]X[2]}
    ID & Estimación & Prioridad  & Tema &  Dependencias \\
    11  & 35 horas & 2  & Producción & 1  \\

    Puesta en producción segura \\

    \textbf{Descripción}:
    Como usuario quiero poder acceder a la plataforma de manera segura para realizar todo lo comentado hasta ahora \\

    \textbf{Criterios de validación}:
    \begin{itemize}
        \item El acceso web tiene que ser por HTTPS (certificado válido).
        \item El servidor debe estar securizado ante accesos externos.
        \item Los servicios deben auto-arrancarse en caso de caída.
        \item Existe un registro de lo sucedido a cada secreto (auditoría).
    \end{itemize} \\
\end{requisitostbl}


\subsection{Mapa de las historias de usuario}

Dado que las historias de usuario pueden suponer la agrupación de distintas tareas técnicas que son más pequeñas, es conveniente crear un mapa que englobe todas las tareas a realizar.

Para la organización de este mapa se pueden identificar los siguientes apartados:
\begin{itemize}
    \item Los \textbf{temas} son las características a alto nivel que va a tener la aplicación, y coinciden con el tema de las historias de usuario. En el mapa se ha diferenciado en color verde.
    \item Los \textbf{epics}, en color azul, son características que normalmente no pueden realizarse en un \textit{sprint}, que a veces pueden ser divididas en distintas historias de usuarios, y que engloban en su mayoría varias tareas.
    \item Las \textbf{tareas}, en este caso en amarillo, son las unidades concretas de trabajo. Se ha tratado que sean lo más cortas posibles y que a nivel técnico sean auto-contenidas y aisladas.
\end{itemize}

El mapa resultante inicial para la gestión del proyecto es el siguiente:

\begin{center}
    \includegraphics[width=\linewidth]{img/kanban.png}
    \captionof{figure}{Mapa de las tareas a realizar}
\end{center}

\subsection{\textit{Sprints} generados}

A la par que se ha ido creando el mapa anterior, se han diferenciado diferentes \textit{sprints} con las tareas que deberían completarse para cada uno de ellos.

Bien es cierto que aunque haya sido realizada esa primera estimación, se podrían efectuar modificaciones en caso de que fuese necesario.

Para el primer \textit{sprint} se han identificado las siguientes tareas que deben realizarse:

\begin{center}
    \includegraphics[width=\linewidth]{img/sprint1.png}
    \captionof{figure}{Tareas a realizar en el \textit{Sprint} 1}
\end{center}

Tal como se puede ver, las tareas están representadas con su prioridad dentro del \textit{sprint}, para así también tener claro qué tareas deben tratar de terminarse antes que otras.

Al terminar este primer \textit{sprint}, tendremos la base del proyecto terminada, y podríamos mostrarle al cliente una primera versión con unas funcionalidades básicas.

De esta manera, podemos obtener \textit{feedback} de lo realizado, con una base ya terminada. Ese \textit{feedback} podremos utilizarlo como retroalimentación para el siguiente \textit{sprint}, ya sea para realizar modificaciones de lo ya realizado (hacer cambios nuevos que el cliente pida) o para utilizarlo para las tareas ya planificadas.

Para llevar una mejor gestión de las tareas expuestas en este apartado, lo ideal es utilizar un software especializado creado con dicho objetivo, como el que vamos a analizar a continuación.

\section{Gestión de tareas con Trello}

\href{https://trello.com/}{Trello} es una aplicación web, creada por la empresa \href{https://www.atlassian.com/}{Atlassian}, que nos permite tener un tablero \href{https://en.wikipedia.org/wiki/Kanban_(development)}{Kanban} donde podremos añadir las tareas de nuestro proyecto, e ir moviéndolas entre distintas etapas.

Aunque cuenta con una versión de suscripción, la versión gratuita cuenta con las características suficientes como para poder ser utilizado para la gestión de proyectos sin ningún problema.

Una vez definidas las tareas, tal como se ha visto anteriormente, se han añadido a Trello, en el que se han creado las siguientes columnas, o fases de desarrollo:

\begin{itemize}
    \item \textbf{Tareas pendientes}: donde situaremos las tareas que se deben realizar para llevar a cabo el proyecto.

\end{itemize}

\begin{minipage}{0.60\linewidth}
    \begin{itemize}
        \item \textbf{Desarrollo}: Son las tareas que se han empezado a desarrollar.
        \item \textbf{Testing}: Son las tareas que se dan por terminadas en el desarrollo y que pasan a una fase de testeo. En algunos casos puede ser directamente testeado por el cliente para dar su feedback (por ejemplo para el interfaz web), aunque lo habitual es que el testeo lo realice otra persona del proyecto (que no haya sido quien ha hecho el desarrollo).

        En caso de que una tarea no pase esta fase, se anotará los inconvenientes y volverá a la columna de \textbf{desarrollo}.
        \item \textbf{Producción}: Una vez el \textit{testing} ha terminado, se puede dar por terminada la tarea y podrá ser incluida en producción.
    \end{itemize}

    Tal como se puede ver en la imagen, cada tarea se representa como una pequeña tarjeta, apareciendo como un listado de todas ellas en la columna correspondiente en las que están situadas (en este caso todavía en la lista de tareas pendientes).
\end{minipage}
\hfill
\begin{minipage}{0.32\linewidth}
    \includegraphics[width=\linewidth]{img/tareas.png}
    \captionof{figure}{Tareas pendientes en Trello}
\end{minipage}

%\vspace{10pt}


A cada una de las tareas se les ha asignado un color teniendo en cuenta el tema al que pertenecen, y varias etiquetas que representan el \textit{sprint} y la prioridad dentro del \textit{sprint}. De esta manera, Trello nos permitirá realizar búsquedas o visualizar las tareas con la etiqueta que nos interese.


\begin{minipage}{0.48\linewidth}
    \includegraphics[width=\linewidth]{img/tarea.png}
    \captionof{figure}{Detalle de la tarea}
\end{minipage}
\hfill
\begin{minipage}{0.40\linewidth}
    \includegraphics[width=\linewidth]{img/tarea1.png}
    \captionof{figure}{Ítems de la tarea}
\end{minipage}

Por último, cada tarea puede tener un listado de pequeños ítems. Estos han sido detallados para que a la hora de desarrollar se tenga en cuenta qué es lo que se debe conseguir, y de esta manera dar por finalizada la tarea.

Tras terminar la tarea, se pasará a la columna \textbf{testing}, para de esta manera, tal como se ha dicho previamente, comenzar con la fase de testeo.



\chapter{Resultados obtenidos}

A continuación se va a exponer los pasos realizados para el diseño y la creación de la aplicación teniendo en cuenta los requisitos del proyecto y las historias de usuario detalladas previamente. Se van a diferenciar distintos apartados generales:

\begin{itemize}
    \item Diseño del \textbf{interfaz}
    \item \textbf{Angular}: Aspectos destacables de la programación
    \item \textbf{Características y uso} de la aplicación
\end{itemize}



\section{Diseño del interfaz}

El interfaz web de la aplicación es el punto de entrada del usuario, por lo que es importante dedicarle el esfuerzo que se merece ya que puede determinar el éxito o el fracaso en el uso de la aplicación.

A la hora de hacer el diseño, se han tenido en cuenta:
\begin{itemize}
    \item \textbf{Facilidad de uso}: Es importante que la aplicación tenga una buena usabilidad, para que el usuario final no requiera de ningún esfuerzo a la hora de utilizarla.
    \item \textbf{Diseño funcional}: Continuando con el punto anterior, se ha tratado de realizar un diseño funcional, para que cada parte del interfaz se sepa para qué va a ser utilizado. Para facilitarlo, se ha trabajado tratando de conseguir un diseño \textbf{minimalista}.
    \item Se ha realizado un \textbf{diseño adaptable a la pantalla} (también conocido como diseño \textit{responsive}). para facilitar el uso de la aplicación en dispositivos móviles.
\end{itemize}


\subsection{Primer boceto}
Para la creación del interfaz se decidió crear un primer prototipo a boli en un cuaderno, tratando de plasmar las ideas generales, y la colocación inicial de los elementos con los que los usuarios van a interactuar.

Este primer boceto sirve para asentar las historias de usuario y las distintas acciones que los usuarios van a poder realizar en la aplicación final. Es una manera rápida para empezar a tomar consciencia del proyecto y cómo queremos encauzar el interfaz.

\vspace{-20pt}
\begin{center}
    \includegraphics[angle=90,width=0.9\linewidth]{img/boceto.pdf}
    \vspace{-20pt}
    \captionof{figure}{Boceto del interfaz}
\end{center}

En el boceto existen zonas diferenciadas hechas en color azul, donde también aparece en color rojo el nombre para cada zona. Ya en este primer boceto se estaban identificando lo que posteriormente serán distintos componentes independientes.

En rojo, en la parte central también aparece cómo sería el login, que daría paso al interfaz general. Por último, se pueden diferenciar, también en la parte central, nombrados como “Opt1” y “Opt2”, las vistas que dependerán de qué estemos haciendo: si navegamos por los secretos o si estamos visualizando/editano un secreto.

Tal como se puede ver, ya en este primer boceto se tenía una idea bastante clara de cómo debía ser a grandes rasgos el interfaz final.

\subsection{Creación inicial del interfaz con un \textit{framework} CSS}
Una vez se tuvieron claro los aspectos generales del boceto, se realizó un primer prototipo haciendo uso de un \textit{framework CSS}.

El \textit{framework} elegido ha sido \href{https://getbootstrap.com/}{Bootstrap}, ya que va a permitir crear un diseño funcional, de manera sencilla, y dispone de distintas características que  van a resultar útiles durante el uso de la aplicación (como por ejemplo los “\textit{modal}” o el “\textit{breadcrumb}”).

\begin{center}
    \includegraphics[frame,width=\linewidth]{img/boceto2.png}
    \vspace{-20pt}
    \captionof{figure}{Boceto del interfaz con Bootstrap y datos estáticos}
\end{center}

Bootstrap ha permitido crear un código sencillo, pero visualmente atractivo, que posteriormente será reutilizado para el interfaz final.


\subsection{Del prototipo al código real}

Tal como se ha comentado previamente, Bootstrap permitió crear crear el prototipo del interfaz de manera rápida y sencilla. Aunque el aspecto visual a grandes rasgos no ha variado en exceso, a medida que se iba avanzando en el proyecto se han ido realizando pequeñas modificaciones.

El interfaz final se puede descomponer en distintos apartados que cumplen con distintas funcionalidades. La pantalla de \textit{\textbf{login}} nos permite introducir el usuario y la contraseña del mismo para acceder a la aplicación.


\begin{center}
    \includegraphics[width=0.4\linewidth]{img/login.png}
    \hfil
    \includegraphics[width=0.4\linewidth]{img/login2.png}
%    \vspace{-10pt}
    \captionof{figure}{Interfaz para el \textit{login}}
\end{center}

En esta ventana de \textit{login} se ha añadido una pestaña extra con el símbolo del engranaje que permite modificar la URL de conexión al que nos queremos conectar

Tras realizar la autenticación la aplicación nos llevará al directorio principal donde podremos navegar a través del interfaz para acceder a los secretos que nos interese.

\begin{center}
    \includegraphics[frame,width=\linewidth]{img/interfaz1.png}
    %    \vspace{-10pt}
    \captionof{figure}{Interfaz principal de la aplicación}
\end{center}

Al acceder al secreto que nos interesa, el interfaz nos muestra el documento descifrado junto con unos botones con los que realizar ciertas acciones sobre los secretos.

\begin{center}
    \includegraphics[frame,width=\linewidth]{img/interfaz2.png}
    %    \vspace{-10pt}
    \captionof{figure}{Interfaz al visualizar un secreto}
\end{center}



\subsection{Interfaz \textit{responsive}}

Para facilitar el uso de la aplicación en distintos tipos de pantallas, se ha hecho que el interfaz se adapte al tamaño donde se está visualizando la aplicación. Esto va a permitir que el usuario pueda acceder a través de su teléfono móvil, lo que puede ser útil en entornos de empresa donde se tengan que realizar salidas a clientes.

\begin{center}
    \includegraphics[width=0.3\linewidth]{img/responsive1.png}
    \hfill
    \includegraphics[width=0.3\linewidth]{img/responsive2.png}
    \hfill
    \includegraphics[width=0.3\linewidth]{img/responsive3.png}

    \captionof{figure}{Interfaz en modo \textit{responsive}}
\end{center}

Tal como se puede ver, para mejorar la navegación, el interfaz que muestra en modo árbol la jerarquía de secretos se oculta, de esta manera navegamos a través de los secretos como un explorador de archivos.


\section{Características y uso de la aplicación}

Teniendo en cuenta los requisitos planteados al inicio del proyecto, y teniendo en cuenta que la aplicación debe ser fácil de utilizar, se ha hecho que las distintas características que dispone la aplicación y el uso de la misma recuerde a aplicaciones que los usuarios ya hayan utilizado.

A continuación se van a detallar distintas características de la aplicación y cómo se crearon pensando en el usuario final.


\subsection{Creación de un secreto nuevo}

La creación de secretos es la característica principal de la aplicación, ya que con ello se va a conseguir guardar la información sensible que nos interese.


{
\begin{minipage}{0.1\linewidth}
    \includegraphics[width=\linewidth]{img/new.png}
\end{minipage}
\hspace{0.5cm}
\begin{minipage}{0.9\linewidth}
    En la cabecera de la aplicación existe un icono que nos va a permitir crear un nuevo secreto en el sistema.

    Al hacer \textit{click} sobre el icono, nos aparecerá un \textit{modal} (una pequeña ventana emergente en el interfaz) para preguntarnos por la ruta donde queremos crear el secreto.
\end{minipage}
}

Al aparecer el \textit{modal} nos aparecerá por defecto la ruta en la que nos encontrábamos en ese instante.

\begin{center}
    \includegraphics[width=0.7\linewidth]{img/new_secret.png}
    \captionof{figure}{\textit{Modal} para crear un secreto}
\end{center}

Si en la ruta que ponemos termina en “/” el botón para crear el secreto permanecerá deshabilitado, ya que una “barra” al final del nombre identifica (al igual que sucede en sistemas UNIX) un directorio.

Al hacer \textit{click} en el botón de creación del secreto la aplicación nos creará un secreto nuevo en la aplicación y nos llevará a la ruta donde lo hemos creado.

\subsection{Acciones sobre secretos}

Una vez creado el secreto, o una vez que navegamos a un secreto creado previamente, podremos realizar distintas acciones sobre el mismo. Estas acciones dependerán de los permisos que se tengan sobre el secreto, pero a continuación se van a detallar todos ellos.

Las acciones que podremos realizar se identifican a través de  distintos iconos que nos aparecen en el interfaz y que se van a detallar a continuación.


\subsubsection*{Editar secreto}
{
\begin{minipage}{0.1\linewidth}
    \includegraphics[width=\linewidth]{img/edit.png}
\end{minipage}
\hspace{0.5cm}
\begin{minipage}{0.9\linewidth}
    Una de las acciones principales es la de \textbf{editar el secreto}, ya sea para añadir, modificar o borrar información en el mismo. Como se verá más adelante, esta edición se hará a través de un editor completo \textit{WYSIWYG}.
\end{minipage}
}

\subsubsection*{Imprimir secreto}
{
\begin{minipage}{0.1\linewidth}
    \includegraphics[width=\linewidth]{img/print.png}
\end{minipage}
\hspace{0.5cm}
\begin{minipage}{0.9\linewidth}
    En algunas ocasiones puede ser interesante \textbf{imprimir la documentación guardada}. Este icono abrirá el asistente para imprimir del navegador y nos mostrará una previsualización.
\end{minipage}
}

A la hora de imprimir, hay ciertas partes del interfaz que desaparecen para mejorar la impresión. La vista en árbol y los botones de acción del secreto no se imprimirán, y la cabecera también sufre una pequeña modificación.


\subsubsection*{Visualizar histórico del secreto}
{
\begin{minipage}{0.1\linewidth}
    \includegraphics[width=\linewidth]{img/calendar.png}
\end{minipage}
\hspace{0.5cm}
\begin{minipage}{0.9\linewidth}
    Como más adelante se explicará, cuando un secreto es modificado \textbf{se guarda un histórico} del mismo. A través de este botón podremos ver las distintas versiones del mismo e ir a una versión anterior.
\end{minipage}
}

\begin{center}
    \includegraphics[width=0.6\linewidth]{img/history.png}
    \captionof{figure}{Histórico de un secreto.}
\end{center}

De esta manera, podremos ver cualquier cambio que haya sufrido el secreto a lo largo del tiempo. Al hacer \textit{click} en alguno de los enlaces para visualizar alguna versión anterior, nos aparecerá un mensaje encima del secreto para indicarlo:

\begin{center}
    \includegraphics[width=0.6\linewidth]{img/history2.png}
    \captionof{figure}{Mensaje indicando la versión del secreto.}
\end{center}

\subsubsection*{Desbloquear secreto}
{
\begin{minipage}{0.1\linewidth}
    \includegraphics[width=\linewidth]{img/unlock.png}
\end{minipage}
\hspace{0.5cm}
\begin{minipage}{0.9\linewidth}
    Cuando un usuario entra a modificar un secreto, este entra en modo “bloqueado”. Si, por alguna razón, abandona la aplicación antes de guardar los cambios el secreto se mantiene en dicho estado. A través de este botón \textbf{un usuario administrador podrá desbloquear un secreto}.
\end{minipage}
}

\begin{center}
    \includegraphics[width=0.8\linewidth]{img/unlock_warning.png}
    \captionof{figure}{Aviso del estado bloqueado de un secreto}
\end{center}


\subsubsection*{Borrar secreto}
{
\begin{minipage}{0.1\linewidth}
    \includegraphics[width=\linewidth]{img/delete.png}
\end{minipage}
\hspace{0.5cm}
\begin{minipage}{0.9\linewidth}
    Un administrador, o usuario que tenga permisos para ello, podrá borrar un secreto con todas las consecuencias que ello conlleva: perder los datos, el histórico, no poder recuperarlo...
\end{minipage}
}

Debido a que es una acción que no se va a poder deshacer, se requerirá que el usuario realice una confirmación antes de que el secreto sea borrado.

\begin{center}
    \includegraphics[width=0.6\linewidth]{img/delete_warning.png}
    \captionof{figure}{Confirmación al borrar un secreto}
\end{center}

Una vez el usuario haga \textit{click} en el botón de la confirmación, el secreto se borrará y la aplicación redirigirá al usuario al directorio superior de donde se encontraba el secreto borrado.

\subsection{Editar un secreto}

Dado que esta acción es la más importante de la aplicación, se va a profundizar a continuación en las características del sistema de edición de secretos.

Tal como se ha visto en el apartado anterior, para editar un secreto existe un botón dedicado a tal acción. Al hacer \textit{click} sobre él, los botones de acción se ocultan y aparece el editor con los datos del secreto:

\begin{center}
    \includegraphics[width=\linewidth]{img/editor.png}
    \captionof{figure}{Editor \textit{WYSIWYG}}
\end{center}

El editor elegido es \href{https://github.com/nhn/tui.editor}{Tui.editor} ya que es un proyecto con \href{https://es.wikipedia.org/wiki/Licencia_MIT}{licencia MIT}, creado en el lenguaje \href{https://es.wikipedia.org/wiki/TypeScript}{Typescript} y que cumple con los requisitos para el proyecto.

Entre las características que tiene el editor, se pueden destacar:

\begin{itemize}
    \item Interfaz \textbf{WYSIWYG} (\textit{what you see is what you get}) que permite guardar el documento generado en formato \href{https://es.wikipedia.org/wiki/Markdown}{Markdown}.
    \item Permite añadir imágenes arrastrándolas sobre el interfaz.
    \item Se puede expandir las funcionalidades a través de \textit{\textbf{plugins}}. Para este proyecto se están utilizando dos:
    \begin{itemize}
        \item Posibilidad de añadir \href{https://github.com/nhn/tui.editor/tree/master/plugins/color-syntax}{color al texto}, expandiendo las posibilidades de Markdown, ya que por defecto este sistema no lo permite.
        \item Mejoras en el \href{https://github.com/nhn/tui.editor/tree/master/plugins/color-syntax}{resaltado de sintaxis} a la hora de añadir bloques de código fuente o de configuración.

        Esto resulta es muy útil cuando se necesita guardar en la documentación configuración de algún tipo o fragmentos de código fuente. El plugin hace uso de \href{https://prismjs.com/}{PrismJS} que soporta 297 lenguajes.
        \begin{center}
            \includegraphics[frame,width=0.7\linewidth]{img/editor_example.png}
            \captionof{figure}{Ejemplo de resaltado de sintaxis}
        \end{center}
    \end{itemize}

    \item Posibilidad de añadir iconos propios a la barra de herramientas del editor. Esta característica ha sido clave ya que era necesario poder crear al menos un botón para guardar y cerrar el secreto.

    El icono añadido tiene el siguiente aspecto \includegraphics[width=0.8cm]{img/save_exit.png}. Al hacer \textit{click} sobre él, el secreto será guardado, el editor se cerrará y la aplicación nos redirigirá para que visualicemos el nuevo aspecto del secreto recién editado.
\end{itemize}

\subsection{Navegación por los secretos}

A la hora de navegar por las contraseñas y documentación se ha hecho uso de un sistema jerárquico que simula las carpetas de un explorador de archivos de cualquier sistema operativo actual.

Los usuarios están acostumbrados a lidiar con cualquier tipo de ficheros y a ordenarlos en sus sistemas de almacenamiento, haciendo uso de una jerarquía decidida previamente.

Dado que la aplicación se ha creado teniendo en cuenta las exigencias de una empresa para lidiar con contraseñas y documentación, se creía conveniente simular este sistema.

Para ello, se han creado 3 sistemas diferenciados en el interfaz, que funcionan en sincronía. Esto quiere decir, que al utilizar cualquier de ellos, hará que se actualice el resto.

\subsubsection*{Vista de árbol}
Es una visualización jerárquica de la información, que se ha situado en el lado izquierdo del interfaz. Este apartado visual siempre será visible en una pantalla de grandes dimensiones, como las de un ordenador de sobremesa.

Si por el contrario, la aplicación se está visualizando a través de un teléfono móvil, el árbol permanecerá oculto, aunque se podrá desplegar a través del botón de la parte superior derecha.

{
\begin{minipage}{0.3\linewidth}
    \includegraphics[width=\linewidth]{img/tree1.png}
\end{minipage}
\hfill
\begin{minipage}{0.3\linewidth}
    \includegraphics[width=\linewidth]{img/tree2.png}
\end{minipage}
\hfill
\begin{minipage}{0.3\linewidth}
    \includegraphics[width=\linewidth]{img/tree3.png}
\end{minipage}
\captionof{figure}{Detalles de la “vista de árbol”}
}

En la vista de árbol se pueden diferenciar:

\begin{itemize}
    \item \textbf{Ramas}: Son un nodo que a su vez contiene otros nodos. En un explorador de ficheros serían los directorios.
    \item \textbf{”Hojas”}: Lo que en un explorador de archivos serían los ficheros.
\end{itemize}

Tal como se ha podido ver previamente, al seleccionar cualquiera de los nodos, se resaltará la opción elegida con un color azul.

\subsubsection*{Explorador de ficheros}

Es el método más habitual a día de hoy para navegar por un sistema de ficheros en los sistemas operativos modernos.

\begin{center}
    \includegraphics[width=0.9\linewidth]{img/browser.png}
    \captionof{figure}{Detalle del explorador de ficheros}
\end{center}

Para identificar el tipo de ficheros que se están visualizando se han identificado mediante distintos iconos. Los iconos utilizados son:

{
\begin{minipage}{0.1\linewidth}
    \includegraphics[width=\linewidth]{img/folder.png}
\end{minipage}
\hspace{0.5cm}
\begin{minipage}{0.9\linewidth}
    Es un directorio, y al igual que sucede en un sistema de ficheros, puede contener a su vez otros ficheros y/o directorios.

    Al hacer \textit{click} sobre él, se visualizará su contenido. Esto afectará a la vista de árbol, que se actualizará para mostrar el directorio desplegado.
\end{minipage}
}


{
    \begin{minipage}{0.1\linewidth}
        \includegraphics[width=\linewidth]{img/secret.png}
    \end{minipage}
    \hspace{0.5cm}
    \begin{minipage}{0.9\linewidth}
        Es un secreto creado y editado a través de la aplicación. Este icono nos identifica que al hacer \textit{click} visualizaremos su contenido, y posteriormente podremos realizar otras acciones como editarlo, imprimirlo, borrarlo...
    \end{minipage}
}


{
    \begin{minipage}{0.1\linewidth}
        \includegraphics[width=\linewidth]{img/file.png}
    \end{minipage}
    \hspace{0.5cm}
    \begin{minipage}{0.9\linewidth}
        Existen distintas representaciones de iconos para identificar a ficheros que se han subido a la aplicación para ser cifrados y almacenados en ella.

        Dependiendo del tipo de fichero (identificado mediante la extensión) mostrará el tipo de fichero que es junto al icono. Existen iconos para las extensiones más habituales.
    \end{minipage}
}

Tal como se ha venido explicando hasta ahora, al simular algo que el usuario ya conoce se va a conseguir que el usuario utilice la aplicación para navegar y guardar la información, consiguiendo que la seguridad de la empresa mejore.


\subsubsection*{Navegación “miga de pan”}

La “miga de pan” (en inglés \textit{breadcrumb}) es una técnica de navegación que se utiliza en distintos tipos de interfaces gráficas de usuario para mostrar el camino recorrido.

Dependiendo de si estamos navegando en un directorio o si estamos en un fichero, al final aparecerá un símbolo “/” o no.

\begin{center}
    \includegraphics[width=0.6\linewidth]{img/breadcrumb1.png}
    \includegraphics[width=0.6\linewidth]{img/breadcrumb2.png}
    \captionof{figure}{Detalle de la “miga de pan”}
\end{center}

Para poder volver sobre nuestros pasos, o para ir hacia arriba en la jerarquía, cada apartado de la “miga de pan” es un enlace, por lo que al hacer \textit{click} sobre cualquiera de ellos nos llevará a su correspondiente directorio.


\subsection{Subir un fichero al sistema}

Dado que la creación y edición de secretos puede resultar limitante, la aplicación permite la posibilidad de subir un fichero para que sea guardado y securizado.


{
    \begin{minipage}{0.1\linewidth}
        \includegraphics[width=\linewidth]{img/upload.png}
    \end{minipage}
    \hspace{0.5cm}
    \begin{minipage}{0.9\linewidth}
        En la cabecera de la aplicación existe un icono que nos va a permitir subir a la aplicación un fichero de cualquier tipo.

        Al hacer \textit{click} sobre el icono, nos aparecerá un \textit{modal} para preguntarnos por la ruta donde queremos guardar el fichero y para que elijamos el fichero de nuestro sistema.
    \end{minipage}
}

\begin{center}
    \includegraphics[width=0.7\linewidth]{img/upload_modal.png}
    \captionof{figure}{\textit{Modal} para subir un fichero}
\end{center}

Al igual que sucedía con la ventana para crear un secreto, este \textit{modal} también cuenta con un sistema de validación que no activará el botón hasta que no se elija del sistema un fichero o una ruta correcta.

Una vez la aplicación ha guardado el fichero de forma segura se nos redirigirá a la ruta donde se ha guardado. Para poder visualizar el fichero sin tener que descargarlo, el sistema detecta (a través del “\href{https://en.wikipedia.org/wiki/Media_type}{MIME Type}”) si un fichero que se puede mostrar, y de esta manera lo visualizará.

Los ficheros que automáticamente se muestran son:

\begin{itemize}
    \item Ficheros \textbf{PDF}: El fichero se abrirá a través del lector de documentos del navegador, lo que posibilita su visualización o descarga.
    \item Ficheros de tipo \textbf{imagen}: Si el “MIME Type” es de tipo \textbf{data:image}, la aplicación creará un elemento en el HTML para poder visualizarlo.
\end{itemize}

\begin{center}
    \includegraphics[frame,width=0.7\linewidth]{img/upload_download.png}
    \captionof{figure}{Visualización de un fichero subido a la aplicación}
\end{center}

Aparte, y tal como se puede ver en la imagen superior, aparecerá un enlace para poder descargar el fichero subido, lo que permitirá recuperar el archivo sin cifrar.



\section{Angular: Aspectos destacables de la programación}

Tras explicar cómo funciona la aplicación desde el punto de vista del usuario, se va a profundizar en algunos aspectos técnicos dentro de la programación.

Tal como se ha dicho, la aplicación está programada utilizando el \textit{framework} \href{https://angular.io/}{Angular} y por tanto se ha hecho uso del lenguaje \href{https://www.typescriptlang.org/}{Typescript}.

A continuación se van a detallar algunos aspectos importantes y algunas características utilizadas durante la programación de la aplicación.


\subsection{Vistas en componentes separados}

Durante la creación del interfaz gráfico se ha dividido en distintos componentes para de esta manera diferenciar el código de la vista para cada uno de ellos.

\begin{center}
    \includegraphics[frame,width=\linewidth]{img/interfaz2-colored.png}
    \captionof{figure}{Interfaz principal con los componentes coloreados}
\end{center}

En la imagen se han coloreado los distintos componentes que existen a la hora de visualizar un secreto:

\begin{itemize}
    \item \textbf{Cabecera}: La cabecera de la aplicación una vez el usuario se ha logueado.
    \item \textbf{Cajón de búsqueda}: Este componente se reutiliza en dos partes del interfaz: en la cabecera y encima de la vista de árbol de secretos.
    \item \textbf{Vista de árbol}: Este componente se oculta cuando el interfaz está siendo visualizado en una pantalla pequeña.
    \item \textbf{Breadcrumb} o “miga de pan”: Este es el componente que nos visualiza la ruta en la que nos encontramos.
    \item \textbf{Vista del secreto}: Esta es la vista principal cuando se está visualizando el secreto. En caso de que se esté editando el secreto, la vista mostrará el editor.
    \item \textbf{Pie de página}: El pie de página de la aplicación.
\end{itemize}

Aparte de los componentes explicados previamente, existen otro dos:

\begin{itemize}
    \item \textbf{Login}: Es el encargado de mostrar el \textit{modal} del \textit{login} cuando no se está logueado en la aplicación.
    \item \textbf{Explorador de ficheros}: Es el componente que visualiza los secretos como si fuese un explorador de ficheros.
\end{itemize}

\begin{center}
    \includegraphics[width=0.9\linewidth]{img/browser.png}
    \captionof{figure}{Detalle del explorador de ficheros}
\end{center}


Varios de estos componentes están integrados en la vista principal a la hora de iniciar la aplicación, en el fichero \configfile{app.component.html}.

\begin{mycode}{Fichero app.component.html}{html}{}
<app-header></app-header>
<div class="container pt-3 mb-5">
    <div class="row">
        <div id="tree_navbar" class="col-md-3 col-lg-2 d-md-block sidebar
         collapse d-print-none">
            <app-tree></app-tree>
        </div>
        <main class="col-md-9 ms-sm-auto col-lg-10 px-md-4">
            <app-breadcrumb></app-breadcrumb>
            <app-browser></app-browser>
            <router-outlet></router-outlet>
        </main>
    </div>
</div>
<app-footer></app-footer>
\end{mycode}


Dado que los componentes idealmente sólo deben centrarse en la experiencia de usuario, se detallará cómo se ha realizado el paso de información entre ellos.



\subsection{\textit{Services} e inyección de dependencias}

Se ha creado un servicio en Angular, cuya dependencia se ha inyectado en la aplicación para realizar el paso de información entre distintos componentes de la vista, y para la obtención de datos de los secretos a través del \textit{backend} Vault utilizando su \href{https://developer.hashicorp.com/vault/api-docs}{API}.

De esta manera


\subsection{Programación asíncrona: \textit{promises} y \textit{observables}}


{\color{red} Comentar cosas como:

    \begin{itemize}
        \item Creación del árbol de secretos usando “promises”
        \item Versatilidad de Angular a la hora de modularizar partes de la web
        \item Sencillez a la hora de comunicar dichos módulos a través de un servicio común. (ejemplo de alguna parte).
    \end{itemize}

{{\large \textbf{NOTA: ¿Esto debería ir quizá como punto 4.2.4?}}}
}





\chapter{Trabajo futuro}

% TODO: posibles mejoras a añadir:

% TODO: configuración del usuario, forzar actualizar el árbol, mejor visualización de ficheros, permitir exportar secretos a formatos editables (odt/docx)

{\color{red} Añadir posibles características futuras:

    \begin{itemize}
        \item Visualizar los logs de auditoria para cada secreto por parte de los administradores
        \item Administrar los permisos de accesos a través de un apartado nuevo de administración.
        \item Crear “timers” para que si el usuario no utiliza la aplicación en X minutos, se realice un deslogueo de la aplicación y así salvaguardar los datos.
        \item Posibles mejoras en la visualización de secretos (que se puedan ver documentos word/libreoffice a través de librerías externas)
        \item Añadir un TOC (índice, table of contents), teniendo en cuenta la estructura del MarkDown.
        \item Poder enviar por mail un documento del secreto (aunque igual me da tiempo a hacerlo antes de la entrega)
        \item ...
    \end{itemize}
}



\chapter{Conclusiones}
{\color{red} Las conclusiones finales}

\vfill

\pagebreak
\printbibliography[title={Referencias bibliográficas},heading=bibintoc]
\pagebreak

{
    \addcontentsline{toc}{chapter}{\listfigurename}
    % un poco de ñapa para quitar el color a los enlaces
    \hypersetup{linkcolor = black}
    \listoffigures
}

\end{document}
